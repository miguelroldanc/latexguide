%bibliotecas
\documentclass{article}
\usepackage{doc}
\usepackage{graphicx}
\usepackage[spanish]{babel}

%Primera cosa siempre para empezar a escribir
\begin{document}
%Hacer una imagen
%\begin{figure}[t]
%\centering
%\includegraphics[scale=0.5]{}
%\end{figure}
%Empezar un texto
\title{Mi primer documento en latex}
\author{Miguel Ángel Roldán}
\date{}
\maketitle

\pagenumbering{roman}
\tableofcontents
\newpage
\pagenumbering{arabic}
%Secciones enumeradas
\section{Introduccion}
Esto es una prueba para ver como se utiliza latex.
\newline Se utilizar los espacios en blanco.

\section{Methods}%Utilizar referencias
\label{sec0}

\subsection{Stage 1}
\label{sec1}
The first part of the methods

\subsection{Stage 2}
The second part of the methods

%Crear tablas
\section{Tabla}
\begin{tabular}{l|c|c}
Item & Quantity & Price(\$)\\
\hline
Nails & 500 & 0.34 \\
Wooden boards & 100 & 4.00 \\
Bricks & 240 & 11.50
\end{tabular}

\section{Results}
Here are my results. Referring to section \ref{sec0} and \ref{sec1} on page \pageref{sec1}
%Parrafo con ecuaciones
\paragraph{}
This is the last part of the document. Mi primera ecuación es $1+2=3$ y la segunda $$1+2=3$$
\end{document}